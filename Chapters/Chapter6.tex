% Chapter 6

\chapter{Conclusioni} % Main chapter title

\label{Chapter6} % Change X to a consecutive number; for referencing this chapter elsewhere, use \ref{ChapterX}

%----------------------------------------------------------------------------------------
%	SECTION 1
%----------------------------------------------------------------------------------------

\section{Risultati}
Questo progetto si può considerare come il più ambizioso, che io abbia mai realizzato in ambito di Computer Graphics, e la prima volta in cui lo scopo non era semplicemente di testare qualcosa, ma di arrivare ad un prodotto che soddisfacesse dei requisiti ben definiti.
Molte delle tecniche e dei processi usati per realizzare il progetto mi erano, fin'ora, completamente sconosciuti.
Ciò ha richiesto un notevole quantitativo di tempo solo per apprendere tutto ciò che c'era di nuovo, in modo da da farne uso in maniera ottimale.
C'è da dire che tutto ciò che è stato appreso ha portato risultati notevoli nel progetto e, senz'altro, tornerà utile in futuro, qualora dovessi realizzare qualsiasi cosa riguardante la grafica 3D.

Il trailer ottenuto non è nulla di che di per se, ma è stato utile a comprendere molti aspetti teorici che sono stati solo leggermente visti durante il corso.
Inoltre, esso rappresenta un pezzo importante da aggiungere al mio portfolio, in quanto rappresenta molto bene tutte le mie abilità riguardanti la grafica 3D.
Realizzare un prodotto attraversando tutte le fasi di un progetto è stato anche utile a capire come ottimizzare il processo e vedere applicazioni di quanto appreso nel mondo reale.

\section{Lavori futuri}

Come già detto, il trailer realizzato è da considerarsi già un ottimo traguardo. Ovviamente dopo la realizzazione di un trailer il passo successivo è quello di realizzare un intero film o una serie ad episodi.
In questo caso si è già pensato a proseguire con una serie, il cui nome sarebbe \emph{"Space Wanderer"}.

Oltre a continuare la serie, che sarebbe soprattutto un buon allenamento e un'interesse personale, ho intenzione di continuare sulla via della computer grafica con un Master. Ho infatti già intrapreso la via per continuare i miei studi iscrivendomi alla Kungliga Tekniska Högskolan (KTH) con un Master of Science in Computer Science, orientato su Visualization and Interactive Graphics.

Il mio obiettivo a lungo termine è quello di diventare un Technical Animator. Questa posizione è una buona combinazione tra le abilità informatiche come problem solving, applicate alla grafica.
L'obiettivo di un animatore tecnico è infatti quello di aiutare gli artisti a fare il loro lavoro, scollegandoli dagli aspetti tecnici delle grafica e delle animazioni.
In pratica si tratta di applicare concetti di programmazione a oggetti come quello dell'information-hiding nel campo delle animazioni.
Questa è una posizione che mi interesserebbe molto ricoprire, in quanto mi permetterebbe di lavorare con ciò che mi appassiona, come la grafica e le animazioni, pur facendo uso di tutto ciò che ho appreso dall'ingegneria informatica.
