% Chapter 6

\chapter{Conclusioni} % Main chapter title

\label{Chapter6} % Change X to a consecutive number; for referencing this chapter elsewhere, use \ref{ChapterX}

%----------------------------------------------------------------------------------------
%	SECTION 1
%----------------------------------------------------------------------------------------

\section{Risultati}
Il trailer ottenuto non è nulla di che di per se, ma è stato utile a comprendere molti aspetti teorici che sono stati solo leggermente visti durante il corso.

Realizzare un prodotto attraversando tutte le fasi di un progetto è utile a capire anche come ottimizzare il processo e vedere applicazioni di quanto appreso nel mondo reale.
\section{Lavori futuri}

Continuare sulla via della computer grafica: Master of Science in Computer Science, Visualization and Interactive Graphics track at KTH.
Becaome Technical Animator: good combination of Computer Science skills (problem solving) applied to graphics to help artist get their job done without having to learn the technical part behind animation.

Basically hiding information that aren't needed to the user (encapsulation like in OOP)
