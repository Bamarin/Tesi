% Chapter 1

\chapter{Introduzione} % Main chapter title

\label{Chapter1} % For referencing the chapter elsewhere, use \ref{Chapter1} 


% Background, what is known
La computer grafica (CG) gioca un ruolo molto importante nell'ambito delle scienze informatiche. In particolare, negli ultimi anni si è visto un notevole miglioramento nelle tecniche di rendering fino ad arrivare a risultati di foto-realismo nei quali è praticamente impossibile riconoscere se un'immagine sia stata generata artificialmente (CGI), o se si tratti di una foto.
Ovviamente questi progressi non sono limitati alle immagini statiche, anzi, trovano molte più applicazioni in sequenze video, dove queste immagini sono animate. 
Di fatto, i settori che maggiormente fanno uso di CGI sono quello cinematografico e quello videoludico.
\newline

% Aims & Objectives
Quello delle animazioni digitali, è un sotto settore della CG, tuttavia molto ampio. Questa tesi vuole quindi analizzare questo aspetto, in particolare le esistenti tecniche di animazione 3D, e il contesto in cui queste vengono utilizzate. Per quanto riguarda il contesto, come già detto sopra, quelli principali sono due: film e videogiochi.

Da qui deriva la scelta di realizzare un cortometraggio: la differenza principale tra i due settori, per quanto riguarda le animazioni, è che il primo utilizza animazioni \emph{offline}, mentre per il secondo si parla di animazione in \emph{tempo reale}.
Ciò significa che nel primo caso è possibile sapere in anticipo esattamente quali azioni saranno da animare mentre, nel caso dei videogiochi, è l'utente finale a scegliere quali azioni fare ed in che ordine. Ciò rende più difficile progettare le azioni necessarie, che devono quindi essere spezzate in animazioni cicliche e rendere possibile transitare da un'animazione all'altra. Quest'ultimo aspetto complica ulteriormente le animazioni dei videogiochi: è particolarmente difficile transitare da un'animazione all'altra e rendere tale transizione fluida e realistica lo è ancora di più.
Un altro aspetto, che ha favorito la scelta di realizzare un cortometraggio rispetto ad un videogioco, è che, da un punto di vista progettuale, il primo è costituito da poche semplici fasi, contrariamente ai videogiochi che racchiudono anche il \emph{game-design}, ovvero rendere il gioco in sé giocabile. Sintetizzando, le animazioni rappresentano un aspetto molto piccolo nella realizzazione di un videogioco, mentre ricoprono la parte centrale nella realizzazione di film d'animazione.

L'obiettivo di un cortometraggio è quello di raccontare una storia semplice e a se stante che possa quindi essere rappresentata come filmato dalla breve durata.
La creazione di un cortometraggio animato è quindi un processo che coinvolge l'intera pipeline grafica con l'aggiunta dell'ideazione di una storia da raccontare. L'intero processo di produzione può quindi essere suddiviso in 3 fasi principali: \emph{pre-produzione}, \emph{produzione} e \emph{post-produzione} \parencite{roy2014finish}.
\newline

% Spiegazione dettagliata
In questo caso verrà analizzata soltanto la fase di produzione (Capitolo \ref{Chapter5}), a sua volta suddivisa
in modellazione, texturing, rigging, animazione, lighting e rendering. Una descrizione più dettagliata di ogni
sotto-fase verrà fornita nel rispettivo paragrafo.

Tuttavia, prima di illustrare come sia stato realizzato il corto, verranno spiegate le fasi di analisi e
progettazione. Come per ogni progetto infatti, è indispensabile partire da un'attenta analisi (Capitolo \ref{Chapter2}) del problema per capire cosa si vuole realizzare e studiarne la sua fattibilità. Di conseguenza sarà necessario definire una metodologia, a questo proposito viene illustrata la progettazione (Capitolo \ref{Chapter4}), che ha l'obiettivo di studiare \emph{come} il progetto verrà realizzato.
Quest'ultima fase, come già detto, vuole analizzare l'aspetto metodologico del progetto. Per rendere possibile ciò, vengono quindi riportate le varie tecniche di animazione al capitolo precedente (Capitolo \ref{Chapter3}).

In ogni capitolo verrà data un particolare attenzione alla parte relativa alle animazioni. Sono infatti esse, l'oggetto principale di questa tesi, a differenza del corto, che rappresenta il prodotto finale: un'applicazione della teoria che sta dietro ai vari metodi di animazione, in un contesto lavorativo reale. Vorrei quindi sottolineare come questa tesi inverta il ruolo delle animazioni e del corto. Normalmente, le animazioni, ed in particolare i \emph{rig} che permettono di animare un oggetto 3D, sono il mezzo che permette di raggiungere lo scopo (oggetto animato). In questo caso invece, è il corto a diventare un semplice mezzo attraverso il quale è possibile mostrare l'efficacia della di diverse tecniche di animazione.


