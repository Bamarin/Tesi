% Chapter 1

\chapter{Introduzione} % Main chapter title

\label{Chapter1} % For referencing the chapter elsewhere, use \ref{Chapter1} 

% Spiegazione generale
L'obiettivo di un cortometraggio è quello di raccontare una storia semplice e a se stante che possa quindi essere rappresentata come filmato dalla breve durata.

La creazione di un cortometraggio animato è quindi un processo che coinvolge l'intera pipeline grafica con l'aggiunta dell'ideazione di una storia da raccontare. L'intero processo di produzione può quindi essere suddiviso in 3 fasi principali: \emph{pre-produzione}, \emph{produzione} e \emph{post-produzione} \parencite{roy2014finish}.

% Spiegazione dettagliata
In questo caso verrà analizzata soltanto la fase di produzione, a sua volta suddivisa in modellazione, texturing, rigging, animazione, lighting e rendering. Una descrizione più dettagliata di ogni sotto-fase verrà fornita nel rispettivo paragrafo. 
