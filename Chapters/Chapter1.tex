% Chapter 1

\chapter{Introduzione} % Main chapter title

\label{Chapter1} % For referencing the chapter elsewhere, use \ref{Chapter1} 

% Spiegazione generale
L'obiettivo di un cortometraggio è quello di raccontare una storia semplice e a se stante che possa quindi essere rappresentata come filmato dalla breve durata.

La creazione di un cortometraggio animato è quindi un processo che coinvolge l'intera pipeline grafica con l'aggiunta dell'ideazione di una storia da raccontare. L'intero processo di produzione può quindi essere suddiviso in 3 fasi principali: \emph{pre-produzione}, \emph{produzione} e \emph{post-produzione} \parencite{roy2014finish}.

% Spiegazione dettagliata
In questo caso verrà analizzata soltanto la fase di produzione (Capitolo \ref{Chapter5}), a sua volta suddivisa
in modellazione, texturing, rigging, animazione, lighting e rendering. Una descrizione più dettagliata di ogni
sotto-fase verrà fornita nel rispettivo paragrafo.

Tuttavia, prima di illustrare come sia stato realizzato il corto, verranno spiegate le fasi di analisi e
progettazione. Come per ogni progetto infatti, è indispensabile partire da un'attenta analisi (Capitolo \ref{Chapter2}) del problema per capire cosa si vuole realizzare e studiarne la sua fattibilità. Di conseguenza sarà necessario definire una metodologia, a questo proposito viene illustrata la progettazione (Capitolo \ref{Chapter4}), che ha l'obiettivo di studiare \emph{come} il progetto verrà realizzato.
Quest'ultima fase, come già detto, vuole analizzare l'aspetto metodologico del progetto. Per rendere possibile ciò, vengono quindi riportate le varie tecniche di animazione al capitolo precedente (Capitolo \ref{Chapter3}).

In ogni capitolo verrà data un particolare attenzione alla parte relativa alle animazioni. Sono infatti esse, l'oggetto principale di questa tesi, a differenza del corto, che rappresenta il prodotto finale: un'applicazione della teoria che sta dietro ai vari metodi di animazione, in un contesto lavorativo reale. Vorrei quindi sottolineare come questa tesi inverta il ruolo delle animazioni e del corto. Normalmente, le animazioni, ed in particolare i \emph{rig} che permettono di animare un oggetto 3D, sono il mezzo che permette di raggiungere lo scopo (oggetto animato). In questo caso invece, è il corto a diventare un semplice mezzo attraverso il quale è possibile mostrare l'efficacia della di diverse tecniche di animazione.



