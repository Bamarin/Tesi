% Chapter 4

\chapter{Progettazione} % Main chapter title

\label{Chapter4} % Change X to a consecutive number; for referencing this chapter elsewhere, use \ref{ChapterX}

%----------------------------------------------------------------------------------------
%	SECTION 1
%----------------------------------------------------------------------------------------

\section{Progettazione generale}

Production oriented

create object once, link in many scene instead of copying it.

Armatures allows to animate objects even if they are linked :D

\section{Progettazione delle animazioni}

Rig: strumento che verrà utilizzato da altri

usabilità

importanza di un rig semplice ed efficace per rendere facile il lavoro dell'animatore
se non è semplice da utilizzare per un animatore è inutile.

capire cosa serve (obiettivo) per rendere il rig il più semplice possibile

Trade-off, controllo VS velocità e semplicità d'uso

scelta della giusta rappresentazione di rotazione (\ref{Section3.1}) basata sul numero di DOF per la giuntura

nel caso di rotazione euleriana (max 2 DOF) scegliere l'ordine degl'assi basandosi su quali verrà effettuata la rotazione:
- più usato -> esterno,world/parent aligned (terzo)
- secondo più usato -> interno local aligned (primo)
- meno usato/non usato -> secondo (causa gimbal lock)