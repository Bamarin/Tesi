% Chapter 2

\chapter{Analisi} % Main chapter title

\label{Chapter2} % For referencing the chapter elsewhere, use \ref{Chapter1} 

\section{Requisiti} \label{req}
L'obiettivo di questo progetto è la realizzare un cortometraggio animato in 3D.
Questo dovrà quindi rappresentare una storia. Non ci sono requisiti su ciò che possa essere rappresentato, quindi si è deciso per una storia di nostra immaginazione, senza aggiunta di particolare realismo.

Per la storia ne è stata scelta una scritta precedentemente da Alberto, il che ci ha permesso di concentrarci sulla realizzazione del corto, senza spendere tempo per l'ideazione della storia. Questa scelta ha quindi introdotto alcuni nuovi requisiti: come la realizzazione dei modelli dei personaggi e degli ambienti rappresentati nella storia, e la fedeltà allo storyboard realizzato dal mio collega.

Cosa più importante: il corto dovrà essere realizzato utilizzando diverse tecniche di animazione, ciascuna adatta al proprio dominio, al fine di mostrarne le diverse caratteristiche, i vantaggi e gli ambiti in qui ha senso utilizzarle.

Come requisito aggiuntivo, non funzionale, si è puntato ad avere animazioni quanto più realistiche possibili. Questo al solo scopo di avere un prodotto finale piacevole da guardare, e poter mostrare con orgoglio.

\section{Vincoli} 
Visti i limiti di tempo, sia per la realizzazione del video, che per poterlo esporre. Si è deciso di proseguire con la realizzazione di un trailer. Ciò non differisce molto da quella che era l'idea iniziale, e soddisfa i requisiti menzionati nella sezione precedente (\ref{req} Requisiti). Questa scelta offre anche il vantaggio di poter affiancare scene temporalmente distanti tra loro, e anche in ordine cronologico non lineare, per mostrare scene topiche in un crescendo di drammaticità.

Inoltre, in questo modo è stato anche eliminato il problema di doppiare i personaggi. Infatti in un trailer, è possibile utilizzare una colonna sonora di sottofondo, al posto dei dialoghi, generalmente per nascondere \emph{spoiler} presenti in ciò che i personaggi dicono.

Infine, un vincolo fondamentale, e alla base di ogni processo di realizzazione di qualsiasi film, è quello di seguire fedelmente lo storyborad, fornito dal direttore artistico, in questo caso sempre Alberto Uras.

\section{Organizzazione del lavoro}
Come per ogni progetto, è importante definire una \emph{tabella di marcia} in maniera tale da non dover rifare le cose più volte, e procedere da un passo all'altro con la consapevolezza di sapere a che punto della produzione ci si trova e, attraverso una stima, capire quanto manca al completamento del prodotto.
Il lavoro è stato quindi suddiviso nelle seguenti task:
\begin{itemize}
    \item Realizzazione e posizione delle scene principali, come key-frame fissi.
    \item Creazione di cicli di animazione da utilizzare in diverse scene.
    \item Implementazione delle animazioni cicliche nelle scene che ne necessitano.
    \item Aggiunta di frame \emph{in-between} e correzione dei movimenti.
    \item Animazione dei dialoghi:
    \begin{itemize}
        \item Creazione delle shape-keys per le diverse espressioni facciali.
        \item Alternare le schape-keys per adattarle all'audio del dialogo.
        \item Aggiunta delle animazioni facciali alle scene animate.
    \end{itemize}
\end{itemize}
