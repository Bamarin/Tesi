%%%%%%%%%%%%%%%%%%%%%%%%%%%%%%%%%%%%%%%%%
% Masters/Doctoral Thesis 
% LaTeX Template
% Version 2.5 (27/8/17)
%
% The original template was downloaded from:
% http://www.LaTeXTemplates.com
%
% Version 2.x major modifications by:
% Vel (vel@latextemplates.com)
%
% This template is based on a template by:
% Steve Gunn (http://users.ecs.soton.ac.uk/srg/softwaretools/document/templates/)
% Sunil Patel (http://www.sunilpatel.co.uk/thesis-template/)
%
% Template license:
% CC BY-NC-SA 3.0 (http://creativecommons.org/licenses/by-nc-sa/3.0/)
%
% Revisited and adapted to fit unibo format
% by Leonardo Marini 
%%%%%%%%%%%%%%%%%%%%%%%%%%%%%%%%%%%%%%%%%

%----------------------------------------------------------------------------------------
%	PACKAGES AND OTHER DOCUMENT CONFIGURATIONS
%----------------------------------------------------------------------------------------

\documentclass[
12pt, % The default document font size, options: 10pt, 11pt, 12pt
%oneside, % Two side (alternating margins) for binding by default, uncomment to switch to one side
italian, % ngerman for German
singlespacing, % Single line spacing, alternatives: onehalfspacing or doublespacing
draft, % Uncomment to enable draft mode (no pictures, no links, overfull hboxes indicated)
%nolistspacing, % If the document is onehalfspacing or doublespacing, uncomment this to set spacing in lists to single
%liststotoc, % Uncomment to add the list of figures/tables/etc to the table of contents
%toctotoc, % Uncomment to add the main table of contents to the table of contents
parskip, % Uncomment to add space between paragraphs
%nohyperref, % Uncomment to not load the hyperref package
headsepline, % Uncomment to get a line under the header
%chapterinoneline, % Uncomment to place the chapter title next to the number on one line
%consistentlayout, % Uncomment to change the layout of the declaration, abstract and acknowledgements pages to match the default layout
]{MastersDoctoralThesis} % The class file specifying the document structure

\usepackage[utf8]{inputenc} % Required for inputting international characters
\usepackage[T1]{fontenc} % Output font encoding for international characters

\usepackage{mathpazo} % Use the Palatino font by default

\usepackage[backend=bibtex,style=numeric,natbib=true]{biblatex} % Use the bibtex backend with the authoryear citation style (which resembles APA)

\emergencystretch=1em
\addbibresource{bibliography.bib} % The filename of the bibliography

\usepackage[autostyle=true]{csquotes} % Required to generate language-dependent quotes in the bibliography
\usepackage{comment}
\usepackage{textcomp}
\usepackage{subcaption}
\usepackage{amsmath}

%----------------------------------------------------------------------------------------
%	MARGIN SETTINGS
%----------------------------------------------------------------------------------------

\geometry{
	paper=a4paper, % Change to letterpaper for US letter
	inner=2.5cm, % Inner margin
	outer=3.8cm, % Outer margin
	bindingoffset=.5cm, % Binding offset
	top=1.5cm, % Top margin
	bottom=1.5cm, % Bottom margin
	%showframe, % Uncomment to show how the type block is set on the page
}

%----------------------------------------------------------------------------------------
%	THESIS INFORMATION
%----------------------------------------------------------------------------------------

\thesistitle{Tecniche di animazione 3D nella realizzazione di un cortometraggio} % Your thesis title, this is used in the title and abstract, print it elsewhere with \ttitle
\supervisor{Prof. Damiana \textsc{Lazzaro}} % Your supervisor's name, this is used in the title page, print it elsewhere with \supname
\examiner{} % Your examiner's name, this is not currently used anywhere in the template, print it elsewhere with \examname
\degree{Laurea} % Your degree name, this is used in the title page and abstract, print it elsewhere with \degreename
\author{Leonardo \textsc{Marini}} % Your name, this is used in the title page and abstract, print it elsewhere with \authorname
\addresses{} % Your address, this is not currently used anywhere in the template, print it elsewhere with \addressname

\subject{Computer Graphics} % Your subject area, this is not currently used anywhere in the template, print it elsewhere with \subjectname
\keywords{tesi, 3D, graphics, cortometraggio, tecniche, animazione} % Keywords for your thesis, this is not currently used anywhere in the template, print it elsewhere with \keywordnames
\university{\href{https://www.unibo.it/it}{Alma Mater Studiorum - Università di Bologna}} % Your university's name and URL, this is used in the title page and abstract, print it elsewhere with \univname
\campus{\href{https://www.unibo.it/it/campus-cesena}{Campus di Cesena}} % Your university's name and URL, this is used in the title page and abstract, print it elsewhere with \univname
\department{\href{https://disi.unibo.it/it}{Dipartimento di Informatica --- Scienza e Ingegneria}} % Your department's name and URL, this is used in the title page and abstract, print it elsewhere with \deptname
\group{\href{http://researchgroup.university.com}{Research Group Name}} % Your research group's name and URL, this is used in the title page, print it elsewhere with \groupname
\faculty{\href{https://corsi.unibo.it/laurea/IngegneriaScienzeInformatiche}{Corso di Laurea in Ingegneria e Scienze Informatiche}} % Your faculty's name and URL, this is used in the title page and abstract, print it elsewhere with \facname

\AtBeginDocument{
\hypersetup{pdftitle=\ttitle} % Set the PDF's title to your title
\hypersetup{pdfauthor=\authorname} % Set the PDF's author to your name
\hypersetup{pdfkeywords=\keywordnames} % Set the PDF's keywords to your keywords
\hypersetup{colorlinks=false}
}

\begin{document}

\frontmatter % Use roman page numbering style (i, ii, iii, iv...) for the pre-content pages

\pagestyle{plain} % Default to the plain heading style until the thesis style is called for the body content

%----------------------------------------------------------------------------------------
%	TITLE PAGE
%----------------------------------------------------------------------------------------

\begin{titlepage}
\begin{center}

\vspace*{.06\textheight}
{\scshape \MakeUppercase{\univname}\\\campname\par}\vspace{1.5cm} % University name
\textsc{\deptname\\\facname}\\ % Thesis type

\vfill
\HRule \\[0.4cm] % Horizontal line
{\huge \bfseries \ttitle\par}\vspace{0.4cm} % Thesis title
\HRule \\[1.5cm] % Horizontal line
 
\vfill
\large {Elaborato in \\\textit{\subjectname}}\\[0.3cm] % University requirement text
\vfill

\begin{minipage}[t]{0.4\textwidth}
\begin{flushleft} \large
\emph{Relatore:}\\
\href{https://www.unibo.it/sitoweb/damiana.lazzaro}{\supname} % Supervisor name - remove the \href bracket to remove the link  
\end{flushleft}
\end{minipage}
\begin{minipage}[t]{0.4\textwidth}
\begin{flushright} \large
\emph{Presentata da:} \\
\href{https://www.linkedin.com/in/leonardo-marini-it/}{\authorname} % Author name - remove the \href bracket to remove the link
\end{flushright}
\end{minipage}\\[3cm]
 
\vfill

{\large Anno Accademico 2018 --- 2019}\\[4cm] % Date
%\includegraphics{Logo} % University/department logo - uncomment to place it
 
\vfill
\end{center}
\end{titlepage}

%----------------------------------------------------------------------------------------
%	ABSTRACT PAGE
%----------------------------------------------------------------------------------------

\renewcommand{\abstractname}{Abstract}
\begin{abstract}
\addchaptertocentry{\abstractname} % Add the abstract to the table of contents

Questa tesi vuole mostrare il processo di produzione di un cortometraggio, da un punto di vista progettuale.
Vengono quindi coperte le fasi della produzione, come la modellazione e la realizzazione delle animazioni.
Oltre ad esse sono ricoperte anche alcune fasi precedenti alla produzione, come la progettazione delle animazioni, e del cortometraggio in generale.
Il focus principale sono quindi proprio le animazioni e la teoria dietro di esse.
Infatti per capire come progettare e realizzare animazioni in maniera efficiente, si rende necessario capire come funzionino da un punto di vista scientifico e ingegneristico.
A tale scopo, vengono illustrati alcuni concetti teorici delle animazioni, con alcuni richiami alla matematica, quali matrici e formule trigonometriche, ed altri di programmazione lineare, come il metodo del simplesso per risolvere un problema di ottimizzazione.

La scelta di un cortometraggio deriva dal fatto che questo è il prodotto che meglio rappresenta un lavoro sulle animazioni.
Un altro approccio sarebbe potuto essere quello di realizzare un videogioco, ma in esso ci sono innumerevoli altri aspetti da progettare, e le animazioni non ricoprono altro che una minima frazione del prodotto finale.
La storia dietro a questo progetto, così come tutti i personaggi, sono opera della creatività di Alberto Uras, caro amico con cui ho avuto il piacere di lavorare a questo progetto.

In questo progetto entrambi abbiamo preso parte in tutte le fasi, ognuno di noi ha però avuto più influenza su alcune di esse, nel mio caso le animazioni.
Per progettarle e realizzarle è stato fatto uso tecniche che mirano a semplificare il lavoro dell'animatore.
Il risultato non è da considerare sorprendente: sono state utilizzate tecniche già da tempo consolidate.
Il punto non è infatti quello di sperimentare nuove tecniche in assoluto. Ciò nonostante, le tecniche che sono state utilizzare erano comunque relativamente nuove in quanto nessuno di noi le aveva mai messe in pratica.

\end{abstract}

%----------------------------------------------------------------------------------------
%	ACKNOWLEDGEMENTS
%----------------------------------------------------------------------------------------
\begin{comment}

\begin{acknowledgements}
\addchaptertocentry{\acknowledgementname} % Add the acknowledgements to the table of contents
Persone da ringraziare\ldots
\begin{itemize}
    \item relatore
    \item familiari
    \item amici
\end{itemize}

\end{acknowledgements}

\end{comment}

%----------------------------------------------------------------------------------------
%	LIST OF CONTENTS/FIGURES/TABLES PAGES
%----------------------------------------------------------------------------------------

\tableofcontents % Prints the main table of contents

\listoffigures % Prints the list of figures

%\listoftables % Prints the list of tables

%----------------------------------------------------------------------------------------
%	ABBREVIATIONS
%----------------------------------------------------------------------------------------

\begin{abbreviations}{ll} % Include a list of abbreviations (a table of two columns)

%\textbf{LAH} & \textbf{L}ist \textbf{A}bbreviations \textbf{H}ere\\
%\textbf{WSF} & \textbf{W}hat (it) \textbf{S}tands \textbf{F}or\\
%\textbf{TBR} & \textbf{T}o \textbf{B}e \textbf{R}emoved\\
\textbf{GCI} & \textbf{C}omputer \textbf{G}enerated \textbf{I}magery\\
\textbf{GC} & \textbf{C}omputer \textbf{G}raphics\\
\textbf{DOF} & \textbf{D}egree(s) \textbf{O}f \textbf{F}reedom\\
\textbf{FK} & \textbf{F}orward \textbf{K}inematics \\
\textbf{IK} & \textbf{I}nverse \textbf{K}inematics \\


\end{abbreviations}

\begin{comment}
%----------------------------------------------------------------------------------------
%	PHYSICAL CONSTANTS/OTHER DEFINITIONS
%----------------------------------------------------------------------------------------

\begin{constants}{lr@{${}={}$}l} % The list of physical constants is a three column table

% The \SI{}{} command is provided by the siunitx package, see its documentation for instructions on how to use it

Speed of Light & $c_{0}$ & \SI{2.99792458e8}{\meter\per\second} (exact)\\
%Constant Name & $Symbol$ & $Constant Value$ with units\\

\end{constants}

%----------------------------------------------------------------------------------------
%	SYMBOLS
%----------------------------------------------------------------------------------------

\begin{symbols}{lll} % Include a list of Symbols (a three column table)

$a$ & distance & \si{\meter} \\
$P$ & power & \si{\watt} (\si{\joule\per\second}) \\
%Symbol & Name & Unit \\

\addlinespace % Gap to separate the Roman symbols from the Greek

$\omega$ & angular frequency & \si{\radian} \\

\end{symbols}

%----------------------------------------------------------------------------------------
%	DEDICATION
%----------------------------------------------------------------------------------------

\dedicatory{For/Dedicated to/To my\ldots} 

\end{comment}

%----------------------------------------------------------------------------------------
%	THESIS CONTENT - CHAPTERS
%----------------------------------------------------------------------------------------

\mainmatter % Begin numeric (1,2,3...) page numbering

\pagestyle{thesis} % Return the page headers back to the "thesis" style

% Include the chapters of the thesis as separate files from the Chapters folder
% Uncomment the lines as you write the chapters

% Chapter 1

\chapter{Introduzione} % Main chapter title

\label{Chapter1} % For referencing the chapter elsewhere, use \ref{Chapter1} 


% Background, what is known
La computer grafica (CG) gioca un ruolo molto importante nell'ambito delle scienze informatiche. In particolare, negli ultimi anni si è visto un notevole miglioramento nelle tecniche di rendering fino ad arrivare a risultati di foto-realismo nei quali è praticamente impossibile riconoscere se un'immagine sia stata generata artificialmente (CGI), o se si tratti di una foto.
Ovviamente questi progressi non sono limitati alle immagini statiche, anzi, trovano molte più applicazioni in sequenze video, dove queste immagini sono animate. 
Di fatto, i settori che maggiormente fanno uso di CGI sono quello cinematografico e quello videoludico.
\newline

% Aims & Objectives
Quello delle animazioni digitali, è un sotto settore della CG, tuttavia molto ampio. Questa tesi vuole quindi analizzare questo aspetto, in particolare le esistenti tecniche di animazione 3D, e il contesto in cui queste vengono utilizzate. Per quanto riguarda il contesto, come già detto sopra, quelli principali sono due: film e videogiochi.

Da qui deriva la scelta di realizzare un cortometraggio: la differenza principale tra i due settori, per quanto riguarda le animazioni, è che il primo utilizza animazioni \emph{offline}, mentre per il secondo si parla di animazione in \emph{tempo reale}.
Ciò significa che nel primo caso è possibile sapere in anticipo esattamente quali azioni saranno da animare mentre, nel caso dei videogiochi, è l'utente finale a scegliere quali azioni fare ed in che ordine. Ciò rende più difficile progettare le azioni necessarie, che devono quindi essere spezzate in animazioni cicliche e rendere possibile transitare da un'animazione all'altra. Quest'ultimo aspetto complica ulteriormente le animazioni dei videogiochi: è particolarmente difficile transitare da un'animazione all'altra e rendere tale transizione fluida e realistica lo è ancora di più.
Un altro aspetto, che ha favorito la scelta di realizzare un cortometraggio rispetto ad un videogioco, è che, da un punto di vista progettuale, il primo è costituito da poche semplici fasi, contrariamente ai videogiochi che racchiudono anche il \emph{game-design}, ovvero rendere il gioco in sé giocabile. Sintetizzando, le animazioni rappresentano un aspetto molto piccolo nella realizzazione di un videogioco, mentre ricoprono la parte centrale nella realizzazione di film d'animazione.

L'obiettivo di un cortometraggio è quello di raccontare una storia semplice e a se stante che possa quindi essere rappresentata come filmato dalla breve durata.
La creazione di un cortometraggio animato è quindi un processo che coinvolge l'intera pipeline grafica con l'aggiunta dell'ideazione di una storia da raccontare. L'intero processo di produzione può quindi essere suddiviso in 3 fasi principali: \emph{pre-produzione}, \emph{produzione} e \emph{post-produzione} \parencite{roy2014finish}.
\newline

% Spiegazione dettagliata
In questo caso le fasi differiscono leggermente da quelle proposte da K. Roy: modellazione e texturing qui vengono trattate come parte della produzione del trailer. Soltanto il rigging viene anticipato nella progettazione siccome va attentamente studiato in base alle animazioni che si vogliono realizzare. Come descritta a K. Roy la produzione (Capitolo \ref{Chapter5}), comprende anche le fasi di animazione, lighting e rendering. Una descrizione più dettagliata di ogni sotto-fase verrà fornita nel rispettivo paragrafo.

Tuttavia, prima di illustrare come sia stato realizzato il corto, verranno spiegate le fasi di analisi e
progettazione. Come per ogni progetto infatti, è indispensabile partire da un'attenta analisi (Capitolo \ref{Chapter2}) del problema per capire cosa si vuole realizzare e studiarne la sua fattibilità. Di conseguenza sarà necessario definire una metodologia, a questo proposito viene illustrata la progettazione (Capitolo \ref{Chapter4}), che ha l'obiettivo di studiare \emph{come} il progetto verrà realizzato.
Quest'ultima fase, come già detto, vuole analizzare l'aspetto metodologico del progetto. Per rendere possibile ciò, vengono quindi riportate le varie tecniche di animazione al capitolo precedente (Capitolo \ref{Chapter3}).

In ogni capitolo verrà data un particolare attenzione alla parte relativa alle animazioni. Sono infatti esse, l'oggetto principale di questa tesi, a differenza del corto, che rappresenta il prodotto finale: un'applicazione della teoria che sta dietro ai vari metodi di animazione, in un contesto lavorativo reale. Vorrei quindi sottolineare come questa tesi inverta il ruolo delle animazioni e del corto. Normalmente, le animazioni, ed in particolare i \emph{rig} che permettono di animare un oggetto 3D, sono il mezzo che permette di raggiungere lo scopo (oggetto animato). In questo caso invece, è il corto a diventare un semplice mezzo attraverso il quale è possibile mostrare l'efficacia della di diverse tecniche di animazione.



% Chapter 2

\chapter{Analisi} % Main chapter title

\label{Chapter2} % For referencing the chapter elsewhere, use \ref{Chapter1} 

\section{Requisiti} \label{req}
L'obiettivo di questo progetto è la realizzare un cortometraggio animato in 3D.
Questo dovrà quindi rappresentare una storia. Non ci sono requisiti su ciò che possa essere rappresentato, quindi si è deciso per una storia di nostra immaginazione, senza aggiunta di particolare realismo.

Per la storia ne è stata scelta una scritta precedentemente da Alberto, il che ci ha permesso di concentrarci sulla realizzazione del corto, senza spendere tempo per l'ideazione della storia. Questa scelta ha quindi introdotto alcuni nuovi requisiti: come la realizzazione dei modelli dei personaggi e degli ambienti rappresentati nella storia, e la fedeltà allo storyboard realizzato dal mio collega.

Cosa più importante: il corto dovrà essere realizzato utilizzando diverse tecniche di animazione, ciascuna adatta al proprio dominio, al fine di mostrarne le diverse caratteristiche, i vantaggi e gli ambiti in qui ha senso utilizzarle.

Come requisito aggiuntivo, non funzionale, si è puntato ad avere animazioni quanto più realistiche possibili. Questo al solo scopo di avere un prodotto finale piacevole da guardare, e poter mostrare con orgoglio.

\section{Vincoli} 
Visti i limiti di tempo, sia per la realizzazione del video, che per poterlo esporre. Si è deciso di proseguire con la realizzazione di un trailer. Ciò non differisce molto da quella che era l'idea iniziale, e soddisfa i requisiti menzionati nella sezione precedente (\ref{req} Requisiti). Questa scelta offre anche il vantaggio di poter affiancare scene temporalmente distanti tra loro, e anche in ordine cronologico non lineare, per mostrare scene topiche in un crescendo di drammaticità.

Inoltre, in questo modo è stato anche eliminato il problema di doppiare i personaggi. Infatti in un trailer, è possibile utilizzare una colonna sonora di sottofondo, al posto dei dialoghi, generalmente per nascondere \emph{spoiler} presenti in ciò che i personaggi dicono.

Infine, un vincolo fondamentale, e alla base di ogni processo di realizzazione di qualsiasi film, è quello di seguire fedelmente lo storyborad, fornito dal direttore artistico, in questo caso sempre Alberto Uras.

\section{Organizzazione del lavoro}
Come per ogni progetto, è importante definire una \emph{tabella di marcia} in maniera tale da non dover rifare le cose più volte, e procedere da un passo all'altro con la consapevolezza di sapere a che punto della produzione ci si trova e, attraverso una stima, capire quanto manca al completamento del prodotto.
Il lavoro è stato quindi suddiviso nelle seguenti task:
\begin{itemize}
    \item Realizzazione e posizione delle scene principali, come key-frame fissi.
    \item Creazione di cicli di animazione da utilizzare in diverse scene.
    \item Implementazione delle animazioni cicliche nelle scene che ne necessitano.
    \item Aggiunta di frame \emph{in-between} e correzione dei movimenti.
    \item Animazione dei dialoghi:
    \begin{itemize}
        \item Creazione delle shape-keys per le diverse espressioni facciali.
        \item Alternare le schape-keys per adattarle all'audio del dialogo.
        \item Aggiunta delle animazioni facciali alle scene animate.
    \end{itemize}
\end{itemize}
 
% Chapter 3

\chapter{Tecniche di animazione} % Design

\label{Chapter3} % For referencing this chapter elsewhere, use \ref{ChapterX}

Animare significa ``muovere" un modello, altrimenti statico, nel tempo. Questi movimenti sono definiti attraverso delle trasformazioni geometriche (traslazione, rotazione e scalatura).
Di seguito vengono riportate alcune tecniche di animazione 3D, spiegandone le metodologie, i pregi, i difetti e il contesto in cui possono venire utilizzate.


\section{Rappresentazioni di rotazione}
Iniziamo spigando come si possa rappresentare una rotazione. Alcuni di questi concetti sono validi anche per altre trasformazioni, tuttavia prenderò in considerazione solo le rotazioni in quanto sono uno degli aspetti principali nella realizzazione di animazioni --- in particolare di animazioni 3D --- e anche uno dei più complessi.


\subsection{Angolo-asse}
Questo tipo di rappresentazione è senza dubbio la più semplice.
Utilizza 4 valori: 3 per specificare l'asse, ed 1 per l'angolo. In questo modo, con una singola rotazione, è possibile raggiungere qualsiasi orientamento dell'oggetto che si sta ruotando. Così come esiste sempre una linea retta che collega due punti nello spazio, si può pensare ad una rotazione angolo-asse come una singola rotazione che collega due orientamenti.

Questa rappresentazione, è ottima per rotazioni di giunture con un solo DOF, mentre il suo comportamento può diventare contro intuitivo per giunture con più di un DOF. Inoltre anche la rappresentazione dell'asse attraverso 3 componenti numeriche non risulta di facile comprensione.

\subsection{Euleriana}

\begin{figure}[]
\centering
\begin{subfigure}{.5\textwidth}
  \centering
  \includegraphics[width=.9\linewidth]{Figures/euler-1.jpg}
  \caption{Posizione neutra}
  \label{fig:sub1}
\end{subfigure}%
\begin{subfigure}{.5\textwidth}
  \centering
  \includegraphics[width=.9\linewidth]{Figures/euler-2.jpg}
  \caption{Gimbal lock, l'asse X e Z sono allineati}
  \label{fig:sub2}
\end{subfigure}
\decoRule
\caption[Rotazione euleriana]{Rappresentazione di rotazione euleriana attraverso un giroscopio a tre assi}
\label{fig:euler1}
\end{figure}

La più intuitiva di tutte: utilizza 3 assi di rotazione (X, Y, Z) ed il funzionamento è analogo a quello di un giroscopio. ogni asse offre un DOF, quindi sono possibili rotazioni con 3 DOF. Tuttavia sono necessarie due accortezze: 
\begin{enumerate}
    \item ordine degli assi;
    \item gimbal lock problem.
\end{enumerate}
L'ordine degli assi è decisivo, in quanto quello più interno dipende dalla rotazione di quelli esterni. Di conseguenza ruotando gli assi in un ordine diverso da quello specificato porta a risultati diversi da quello atteso. In più, interpolazioni tra diverse orientazioni possono a loro volta risultare differenti da ciò che ci si sarebbe aspettato.

Il problema del gimbal lock \parencite{anticz16}, in italiano blocco cardanico, sorge dall'allineamento di due assi: quello più interno e quello più esterno. Ne deriva che ruotano uno di questi 2 assi si ottiene la stessa rotazione, perdendo quindi un DOF.
È quindi importante scegliere l'ordine degli assi in maniera tale che il primo e il terzo non risultino mai allineati.





\subsection{Quaternaria}

\subsection{Matriciale}
Quest'ultima è la rappresentazione ottimale, in quanto permette di rappresentare anche traslazioni, scalature e altre trasformazioni come \emph{shear}. È, infatti, la rappresentazione che blender utilizza internamente \parencite{blendApi, nat2012rig} poiché quella che offre la maggior flessibilità. L'unico difetto è che, come la rappresentazione quaternaria, non mantiene l'informazione sul percorso della rotazione. Infatti è ancora più simile a una delta-rotazione (i.e. differenza di orientamento), rispetto alla rappresentazione quaternaria poiché copre una rotazione di soli 360\textdegree, rispetto ai 720\textdegree\ delle quaternarie. 



% Chapter 4

\chapter{Progettazione} % Main chapter title

\label{Chapter4} % Change X to a consecutive number; for referencing this chapter elsewhere, use \ref{ChapterX}

Così come nella progettazione di un software si passa dall'analisi alla sua progettazione, prima di svilupparlo. Anche in questo caso, è opportuno progettare l'intero cortometraggio, prima di realizzarlo.
Questo è un passaggio importante poiché non solo permette di capire come il prodotto dovrà essere realizzato, ma permette anche di stabilire delle convenzioni standard (e.g. nomi dei file) da mantenere durante il progetto.
Quest'ultimo aspetto è indispensabile soprattutto nel caso in cui ci siano più persone a lavorare allo stesso progetto.

%----------------------------------------------------------------------------------------
%	SECTION 1
%----------------------------------------------------------------------------------------

\section{Progettazione generale}

Production oriented



create object once, link in many scene instead of copying it.

Armatures allows to animate objects even if they are linked :D

\section{Progettazione delle animazioni}

Quando si tratta di sviluppare qualcosa di complesso come una figura umana, è opportuno progettarla, per trovare un modello di astrazione che la semplifichi, pur mantenendo le proprietà che ci interessano e quindi ci permetta di animarla in maniera efficiente.
Il corpo umano è infatti composto da circa duecento DOF \cite{Parent:2012:CAA:2385444}.
Nonostante ciò, la struttura esteriore è fondamentalmente composta da un'unica mesh.
Quindi, rispetto a quanto visto fin'ora, dove la struttura da animare era un'armatura composta da più ossa, sarà necessario deformare la mesh per adattarla all'armatura sottostante.

Per fare ciò sarà necessario 

definire un \emph{rig}, fondamentalmente 

strumento che verrà utilizzato da altri

usabilità

importanza di un rig semplice ed efficace per rendere facile il lavoro dell'animatore
se non è semplice da utilizzare per un animatore è inutile.

capire cosa serve (obiettivo) per rendere il rig il più semplice possibile

Trade-off, controllo VS velocità e semplicità d'uso

scelta della giusta rappresentazione di rotazione (\ref{Section3.1}) basata sul numero di DOF per la giuntura

nel caso di rotazione euleriana (max 2 DOF) scegliere l'ordine degl'assi basandosi su quali verrà effettuata la rotazione:
- più usato -> esterno,world/parent aligned (terzo)
- secondo più usato -> interno local aligned (primo)
- meno usato/non usato -> secondo (causa gimbal lock)

Animazione straight ahead vs pose-to-pose (mixing between the 2) 
% Chapter 5

\chapter{Produzione} % Main chapter title

\label{Chapter5} % Change X to a consecutive number; for referencing this chapter elsewhere, use \ref{ChapterX}

%----------------------------------------------------------------------------------------
%	SECTION 1
%----------------------------------------------------------------------------------------

\section{Modellazione}

Con modellazione digitale, o modellazione 3D, o semplicemente modellazione, per brevità, si intende la manipolazione di veritci in uno spazio tridimensionale per realizzare solidi più o meno complessi denominati modelli.
Questa è senz'altro la fare più artistica dell'intero processo di produzione, e una di quelle che più mi ha appassionato.

Una volta definiti i concept art, in gran parte realizzati da A. Uras, il primo passo della produzione è appunto quello di realizzare i modelli dei personaggi, dei prop e delle ambientazioni.
Per fare ciò, mi sono avvalso di diverse tecniche di modellazione 3D, che possono essere divise nelle seguenti 2 categorie:
\begin{itemize}
    \item Modellazione hard-surface \cite{hardSurf}.
    \item Scultura digitale\cite{3Dsculpt}.
\end{itemize}
La differenza fondamentale sta nel cosa si vuole realizzare. La prima è ottima per quasi tutto ciò che è realizzato dall'uomo (e.g. macchinari, edifici) ed è stata quindi stata usata nella realizzazione di ambientazioni e prop.
Consiste nel partire da una forma geometrica semplice, ad esempio un cubo o una sfera, ed aggiungere dettagli estrudendo sezioni o utilizzato operatori booleani in combinazione con altre forme geometriche. (Figura \ref{fig:box-model})
Il risultato è un modello dagli spigoli ben definiti, che comunque può avere superfici curve e smussate.
La caratteristica che lo distingue è la simmetria della struttura e la precisione della posizione dei dettagli. 

\begin{figure}
\centering
\begin{subfigure}{.5\textwidth}
  \centering
  \includegraphics[width=.99\linewidth]{Figures/box1.png}
\end{subfigure}%
\begin{subfigure}{.5\textwidth}
  \centering
  \includegraphics[width=.99\linewidth]{Figures/box2.png}
\end{subfigure}
\begin{subfigure}{.5\textwidth}
  \centering
  \includegraphics[width=.99\linewidth]{Figures/box3.png}
\end{subfigure}%
\begin{subfigure}{.5\textwidth}
  \centering
  \includegraphics[width=.99\linewidth]{Figures/box4.png}
\end{subfigure}\\[2ex]
\decoRule
\caption[Modellazione hard-surface]{Esempio di modellazione hard-surface per modellare un'astronave.}
\label{fig:box-model}
\end{figure}

Al contrario, il secondo metodo viene utilizzato per ottenere modelli organici, in pratica tutto ciò che è presente in natura.
Questi modelli sono particolari, poiché andranno deformati per essere animati.
Un esempio lampante, in questo progetto, sono i personaggi.
Nonostante anche in questo caso la simmetria sia fondamentale (ogni personaggio ha 2 braccia e due gambe perfettamente simmetriche), vengono solitamente aggiunti dettagli per rompere la simmetria, siccome in natura nulla è perfettamente simmetrico.
Anche in questo caso si parte solitamente da delle forme geometriche semplici, tuttavia il processo di modellazione è completamente differente, tant'è che spesso si fa distinzione tra modellazione e scultura digitale (come se i due concetti si escludessero a vicenda).
Nella scultura digitale infatti non si usano mai operazioni come l'estrusione di una faccia, di fatto il concetto di faccia non viene neanche utilizzato.
La mesh viene vista come un oggetto compatto, i cui vertici possono essere manipolati attraverso strumenti che emulano le azioni di uno scultore.
Per questo motivo la scultura digitale è anche il metodo di modellazione preferito dagli artisti.
\newline

\begin{figure}
\centering
\includegraphics[width=.8\textwidth]{Figures/bandit-concept}
\decoRule
\caption[Concept art]{Concept art di uno dei porsonaggi}
\label{fig:concept}
\end{figure}
\begin{figure}
\centering
\includegraphics[width=.8\textwidth]{Figures/bandit-head}
\decoRule
\caption[Modello 3D]{Rappresentazione 3D del concept art mostrato in Figura \ref{fig:concept}}
\label{fig:model}
\end{figure}
Affinché un modello si possa definire ben fatto e ultimato, è necessario soddisfare i seguenti criteri.
In primo luogo, deve ben rappresentare in 3D quello che il concept si limitava a mostrare in 2D.
In Figura \ref{fig:concept} e \ref{fig:model} è possibile notare quanto il modello 3D si avvicini al concept originale.
Nonostante, in questo caso, il modello rappresenti molto bene il concept, è possibile notare come alcuni dettagli differiscano.
Questo è dovuto principalmente a due motivi: il primo è che, semplicemente, il concept art serve a dare un'idea di come deve apparire il personaggio, e non vincola quindi a mantenere gli stessi dettagli durante la fase di modellazione.
Il secondo è che in un disegno 2D la forzatura della prospettiva può nascondere dettagli a cui un modello 3D non può evadere.
Per tanto, sono spesso necessarie modifiche poiché non sarebbe possibile realizzare in 3D ciò che era rappresentato in 2D.

\begin{figure}
\centering
\includegraphics[width=.8\textwidth]{Figures/bandit-wire}
\decoRule
\caption[Topologia]{Modello 3D con topologia della mesh visibile}
\label{fig:wire}
\end{figure}
Il secondo criterio, non meno importante del primo, è la \emph{topologia} della mesh.
A dir la verità, questo aspetto non ha alcuna importanza nel caso di modelli statici, mirati allo scopo di essere rappresentati in un render fisso, o stampati in 3D.
Tuttavia, nel caso delle animazioni, questo è un aspetto fondamentale.
Infatti dal posizionamento dei vertici dipende la deformazione della mesh che, com'è già stato detto, è necessaria nelle animazioni dei modelli organici (e.g. personaggi).
Per avere una buona topologia, è fondamentale che le facce siano composte di quattro lati. Questo serve a definire dei percorsi, che attraversano la faccia da un lato a quello opposto e continuano nelle facce adiacenti.
Questi percorsi servono a far si che quando si aggiunge un maggiore livello di dettaglio, la mesh non venga deformata in maniera inaspettata.
Un altra caratteristica importante, per una buona topologia, è quella di avere al massimo quattro spigoli (o archi) che partano da ogni vertice.
Questo non è sempre possibile, tuttavia è necessario che vertici con più di cinque spigoli siano presenti nel minor numero possibile, e siano posizionati in punti strategici dove la mesh verrà difficilmente deformata.
Questo perché se vertice è adiacente a molti vertici, genera anche molti incroci di percorsi che sono da evitare per quanto detto prima.

Infine, un aspetto non poco importante nella definizione qualitativa di un modello, è il numero totale di vertici.
L'obiettivo è quello di avere il minor numero possibile di vertici per poter realizzare un certo livello di dettaglio.
Il motivo anche qui è duplice, seppure i due effetti sono strettamente correlati.
Innanzi tutto, un ridotto numero di vertici permette un \emph{frame-rate} più alto in fase di animazione.
In secondo luogo, in fase di rendering, ogni frame impiegerà meno tempo ad essere renderizzato.

Per mantenere un livello di dettaglio, discretamente alto, è stato fatto uso di una tecnica denominata \emph{baking}, che verrà approfondita nel paragrafo successivo.
Per ora mi limiterò a dire che per ogni modello è stato fatta prima una versione "\emph{low-poly}", ovvero a basso contenuto di poligoni (o vertici, analogamente).
Dopodiché i dettagli sono stati modellati su una copia di questo modello, dopo averne aumentato il numero di vertici suddividendo ogni faccia.
Avere due modelli separati, ci permette di utilizzare il primo nelle animazioni e, in un secondo momento, aggiungere i dettagli "cucinati".

Qui i vantaggi sono molteplici: è possibile iniziare ad animare un modello subito, parallelizzando la fase di animazione a quella di modellazione dei dettagli.
In più come già detto, si avrà un frame-rate più alto in fase di animazione, ed un rendering più veloce una volta ultimate le animazioni.

In Figura \ref{fig:boy} \ref{fig:cap} \ref{fig:lt} e \ref{fig:prop} sono riportati altri esempi di comparazione tra concept e modello 3D dei modelli da me realizzati. Tutti i concept, a parte quello del tenente, sono stati forniti da Alberto Uras. 
\begin{figure}
\centering
\includegraphics[width=.8\textwidth]{Figures/boy}
\decoRule
\caption[Ragazzo]{Modello 3D del volto del ragazzo a confronto con il relativo concept.}
\label{fig:boy}
\end{figure}
\begin{figure}
\centering
\includegraphics[width=.8\textwidth]{Figures/cap}
\decoRule
\caption[Capitano]{Modello 3D del volto del capt. Lawrence a confronto con il relativo concept.}
\label{fig:cap}
\end{figure}
\begin{figure}
\centering
\includegraphics[width=.8\textwidth]{Figures/lt}
\decoRule
\caption[Tenente]{Modello 3D del volto del tenente a confronto con il relativo concept.}
\label{fig:lt}
\end{figure}
\begin{figure}
\centering
\includegraphics[width=.8\textwidth]{Figures/props}
\decoRule
\caption[Prop]{Alcuni modelli 3D dei prop a confronto con il relativo concept.}
\label{fig:prop}
\end{figure}

\newpage
\section{Texturing}

Il processo di texturing serve ad applicare un'immagine alla superficie di un oggetto. Allo scopo di aggiungergli colore e dettagli di rilievo, detti appunto texture.  
Chiaramente i modelli rappresentati nelle figure non possono considerarsi completi senza l'aggiunta di colore.
Uno degli scopi delle texture è proprio questo, esse infatti permettono di mappare un'immagine alla superficie della mesh.
Per fare ciò però, la mesh deve prima essere stesa. Questo è possibile farlo in maniera automatica, il problema è che spesso la topologia deve venire spezzata per poterla stendere su un piano.
In alternativa, per figure complesse, è consigliato farla manualmente, in modo da limitare i ``tagli", che creerebbero discontinuità una volta applicata la texture.

Dopo aver eseguito questa procedura, è possibile creare una texture che si abbini a ciò che è stato ottenuto. Esistono diverse tecniche per fare ciò:
\begin{itemize}
    \item pittura manuale;
    \item generate proceduralmente.
    \item generata dalla geometria della mesh;
\end{itemize}
La prima, come suggerisce il nome, è la più artistica.
Si tratta fondamentalmente di disegnare sul modello.
Questa tecnica, seppure non sia affatto automatizzabile, è la più utilizzata, in quanto spesso non c'è altro modo di aggiungere dettagli irregolari. Essa è ad esempio stata usata per aggiungere il dettaglio ai capelli del capitano, per non dare l'idea che la sua chioma fosse un blocco unico, ma formato da tanti capelli.

Per i capelli di tutti gli altri personaggi, la texture è stata generata proceduralmente. Questo è stato possibile poiché, avendo questi altri, dei capelli lunghi, questi sono stati modellati con l'uso di curve in varie ciocche.
Dopodiché è stato possibile definire la texture in modo che fosse formata da tante linee che partissero da un estremo della curva all'altro.

Ad ogni modo le texture non sono usate solo per dare colore ad un oggetto. Esse sono utilizzate anche per aggiungere dettagli in rilievo.
Attraverso l'uso di particolari immagini, come mappe normali, che utilizzano i colori per rappresentare informazioni come la direzione della superficie, è possibile dare maggiore profondità ad un oggetto.
Questa tecnica è detta "baking", come era già stato menzionato in precedenza, permette di aggiungere dettagli ad un oggetto senza aumentarne il numero di vertici.
Il concetto che ne sta dietro è quello di pre-calcolare, una sola volta i dettagli, e poi applicarli al modello in bassa risoluzione.
In questo caso la texture viene generata direttamente dalla geometria della mesh.

\newpage
\section{Animazione e Rigging}

\begin{figure}
\centering
\begin{subfigure}{.33\textwidth}
  \centering
  \includegraphics[width=\linewidth]{Figures/rig0}
  \caption{Modello senza controlli.\\}
  \vspace{13pt}
  \label{fig:rig0}
\end{subfigure}%
\begin{subfigure}{.33\textwidth}
  \centering
  \includegraphics[width=\linewidth]{Figures/rig1}
  \caption{Armatura o meta-rig.}
  \bigskip
  \label{fig:rig1}
\end{subfigure}%
\begin{subfigure}{.33\textwidth}
  \centering
  \includegraphics[width=\linewidth]{Figures/rig2}
  \caption{Rig avanzato con forme personalizzate.}
  \label{fig:rig2}
\end{subfigure}
\decoRule
\caption[Rig a confronto]{In figura è mostrato il modello di uno dei personaggi, con diversi tipi di controlli per l'animazione.}
\label{fig:rig}
\end{figure}

Rigging è un termine generale che si riferisce all'aggiunta di controlli ad un oggetto, tipicamente allo scopo di animarlo \parencite{blendDoc}.
Consiste nell'assegnare relazioni tra oggetti \parencite{BlendTut}.

Nel Capitolo \ref{Chapter4} è stato visto come progettare un rig orientato all'animazione che bisognerà eseguire.
Nella realizzazione di questi si è partiti definendo il meta-rig, aggiungendo tutte le ossa, dalla radice alle foglie, e associando ciascuna alla rispettiva porzione del modello, per permettere una deformazione soddisfacente di quest'ultimo.
Per generare il rig avanzato, è stato in parte usato rigify \cite{blendDoc}: una tecnologia che permette di automatizzare alcuni processi della realizzazione di un rig.
Purtroppo questa tecnologia è stata scoperta verso la fine del progetto e non è quindi stata utilizzata al massimo. 
Infatti i rig avanzati non sono quasi per nulla stati utilizzati, anche perché, essendo io e Uras già familiari nell'animare direttamente il meta-rig, si è preferito procedere direttamente alla realizzazione delle animazioni.
In Figura \ref{fig:rig} è possibile vedere il rig finito applicato ad un modello.

\subsection{Keyframe e metodi di animazione}
Ora che è possibile animare i modelli deformandoli, si può procedere con la realizzazione delle animazioni, precedentemente descritte al Capitolo \ref{Chapter4}.
Esistono principalmente due metodi di animazione \cite{Williams:2009:ASK:1823185}:
\begin{itemize}
    \item dall'inizio alla fine
    \item da posizione a posizione
\end{itemize}
Il primo è molto semplice: si parte dal primo frame, dove si posizionano all'interno della scena tutti i personaggi e la telecamera. Poi si passa al secondo animando ogni cosa, poi al terzo e così via. 
Questo metodo è molto utilizzato nell'animazione tradizionale, soprattutto nella realizzazione dei flip-book.
Ha il vantaggio di essere naturale: per arrivare in una determinata posizione devo spostarmi facendo un passo alla volta partendo dal primo punto in avanti.
Di conseguenza anche le animazione che ne risultano hanno un aspetto più naturale.
Questo metodo però rende difficile pianificare le azioni future per questo motivo non è ottimale.

Al contrario, animare da posizione a posizione permette di pianificare un'intera scena definendo delle posizioni chiave.
Queste ultime, anche dette \emph{key-frame}, provengono direttamente dallo storyboard e definiscono cosa deve accadere in una scena.
Definite queste posizioni principali, distribuite nella sequenza dei frame della scena, si possono aggiungere i frame di intermezzo tra due posizioni.
Nell'animazione digitale questo metodo è il più efficiente in quanto i frame di intermezzo possono venire automaticamente calcolati come interpolazione delle due posizioni.
Il lato negativo è che le animazioni risultanti non sono molto naturali: spesso è necessario rendere un personaggio più veloce o più lento, per farlo arrivare in un ponto al tempo giusto e rispettare i piani preposti.

Per realizzare tutte le animazioni di questo cortometraggio è stato fatto ovviamente uso di quest'ultimo metodo.
Va aggiunto però che, per rendere le animazioni più naturali, dopo aver definito tutti i frame di intermezzo.
Questi ultimi sono stati modificati con un approccio dal primo verso l'ultimo.
In questo modo l'azione è più naturale, e spezza la linearità dell'interpolazione che, essendo calcolata dal computer, non mostra mai un movimento realistico fin da subito.
È quindi più giusto dire che nella realizzazione del trailer, è stato fatto uso di una combinazione delle due tecniche.

\subsection{Animazioni cicliche}
Inoltre, per le animazioni più comuni, esse sono state rese cicliche, per poterle iterare per la durata necessaria, e riutilizzarle in ogni scena necessaria.
In questo modo le animazioni sono state modularizzate, ovvero spezzate in più azioni ripetibili.
Alcune animazioni come la camminata, infatti, seguono dei pattern ben definiti, costituidi da varie fasi e,
grazie alla loro uniformità è possibile renderle cicliche in maniera tale da non dover rifare la stessa animazione più volte.

Nel caso del ciclo della camminata queste fasi sono dette a supporto singolo e doppio supporto \cite{Parent:2012:CAA:2385444}.
Quest'ultima inizia con la posa di \emph{contatto}, in cui il secondo piede poggia a terra col tallone, e termina con la posizione di passaggio.
Viceversa la fase a singolo supporto inizia dalla posizione di passaggio e termina con la posizione di contatto.
Aggiungendo un solo frame di intermezzo per ogni fase, rispettivamente posa bassa (doppio contatto) e posa alta (contatto singolo), si ottiene un passo completo. Ripetendo il tutto per l'altra gamba l'animazione può considerarsi ultimata. I restanti frame verranno calcolati tramite interpolazione.

La corsa è molto simile alla camminata, ma differisce nella durata delle fasi e nel fatto che, a differenza della camminata, dove almeno un piede poggia sempre a terra, in questo entrambi i piedi non sono mai a contatto col suolo nello stesso momento.
Le due fasi sono quindi chiamate supporto singolo e volo, in cui nessun piede si trova a contatto col terreno.

Un volta ottenuta l'animazione ciclica, è possibile inserirla in ogni scena necessaria.
Nel caso di una camminata, è inoltre necessario spostare il modello mentre cammina.
Il modo migliore per farlo è quello di definire un percorso, che il modello dovrà seguire durante l'animazione.
Il percorso non è altro che una curva, solitamente NURBS o B-spline.
Il vantaggio di usare le curve in questo caso è che il percorso risultante avrà curve molto soffici.
Inoltre è possibile, attraverso i punti di controllo, modificare la curva a posteriori e, con essa, il modello si sposterà di conseguenza.

\subsection{Espressioni e Dialoghi}
Un altro aspetto importante delle animazioni, soprattutto in questo progetto, sono i dialoghi.
Nonostante si sia scelto di non doppiare i dialoghi dei personaggi, ma piuttosto tenere una traccia audio di sottofondo, per aggiungere drammaticità, è comunque stato necessario animare le facce dei personaggi per permettergli di parlare e esprimere sentimenti attraverso espressioni facciali.
Per fare ciò sono state utilizzate due tecnologie: \emph{shape-keys} \cite{blendDoc} e \emph{Rhubarb} \cite{blendRhubarb}.

Siccome la faccia è un concentrato di muscoli servirebbe un rig con centinaia di DOF.
Avere molte ossa per controllare ogni muscolo non è conveniente, poiché sarebbe difficile da animare.
Per semplificare il processo di animazione è conveniente parametrizzare ogni muscolo con una semplice variabile di range $0-1$ dove 0 indica il muscolo rilassato e 1 contratto.
Tuttavia a causa dell'elevato numero di muscoli presenti nella faccia sarebbe comunque difficile animare una singola espressione.
Un modo per ovviare a questo problema e quello di usare appunto delle shape-keys, conosciute anche come blend-shapes, dove ognuna rappresenta una posa della faccia (e.g. triste, felice, arrabbiato, sorpreso...), o meglio dividere ogni posa in aree della faccia e poi parametrizzarle e combinarle in modo che la loro somma sia uguale a 1.
Questo metodo è molto restrittivo rispetto al grado di controllo che ha l'animatore, poiché le espressioni rappresentabili dipendono dalle shape-keys definite precedentemente (e dalle loro combinazioni interpolanti).
Nonostante ciò rappresenta un buon compromesso tra semplicità d'uso e numero di espressioni rappresentabili.

Per quanto riguarda i dialoghi, Rhubarb Lyp Sync, permette di animare la bocca, e le aree limitrofe del viso, attraverso una traccia audio e scritta del dialogo da riprodurre.
Ciò è possibile perché a ogni fonema corrisponde un visema (i.e. espressione della faccia).
Per tanto, è stato sufficiente modellare un set limitato di espressioni e associarle ad il rispettivo fonema, registrare l'audio, e Rhubarb ha fatto il resto.
Come riportato da Parent \cite{Parent:2012:CAA:2385444}, esistono 42 diversi fonemi. Tuttavia molti di questi possono essere rappresentati dallo stesso visema, o da una interpolazione di questi.
Inoltre Williams afferma che non è necessario animare una frase in ogni sua singola sillaba, è sufficiente selezionare quelle più evidenti \cite{Williams:2009:ASK:1823185}. Questo ha permesso di ridurre il numero di shape-keys necessarie per i dialoghi a 9 che sono anche quelle utilizzate da Rhubarb.

\section{Illuminazione}

Quello dell'illuminazione è un passaggio fondamentale per poter renderizzare una scena. Infatti senza luci il risultato sarebbe ovviamente quello di una schermata nera.
Non solo, esistono diversi tipi di luce \cite{lightArt}, e da essi, oltre che dal loro posizionamento, l'aspetto di una scena può cambiare completamente.

Esistono fondamentalmente due categorie di settaggio di luci: per ambienti e per soggetti in primo piano.
Quest'ultimo è quello solitamente utilizzato dai fotografi nei loro studi fotografici: l'obiettivo è quello di creare una luce ad hoc, che risalti le forme del soggetto che si vuole rappresentare.
Questo è molto importante perché nella fotografia, così come nel rendering, si passa da una scena in tre dimensioni a una sua rappresentazione bidimensionale. 
È quindi facile perdere le informazioni di profondità, e le ombre servono proprio a questo. Solitamente per raggiungere questo scopo si utilizza un sistema a tre luci \cite{3Plight}.

L'illuminazione di un ambiente può spesso risultare più complessa. È importante indirizzare la luce per evidenziare cosa si vuole mostrare.
Il tutto è reso complesso dal fatto che ogni luce deve avere un contesto: aggiungere una luce che fluttua a mezz'aria non sarebbe realistico, bisogna che provenga da una fonte luminosa, come ad esempio una lampada.
Tutto comunque dipende dal contesto: non avrebbe senso aggiungere una lampada da tavolo in una scena all'esterno.

Indipendentemente dal tipo di scena che si vuole realizzare, il metodo in cui la luce viene calcolata per mostrare oggetti o oscurare l'ambiente circostante con ombre è sempre lo stesso, e dipende dal motore di renderizzazione.

\section{Renderizzazione}

Il processo di renderizzazione è ciò che permette di proiettano la scena tridimensionale in un’immagine in una finestra contenuta nello schermo bidimensionale \cite{renderingPipelineDLazzaro}.
Per fare ciò, è necessario definire una camera virtuale, attraverso un punto nello spazio tridimensionale che ne definisce la posizione, un vettore che indica la direzione in cui essa è orientata, un angolo che definisce il campo visivo ed infine la profondità oltre la quale non è necessario renderizzare.

Attraverso una serie di trasformazioni, ogni oggetto presente nella scena viene proiettato nel sistema di riferimento della telecamera.
Esiste comunque più di un modo per calcolare queste trasformazioni, due di questi sono rasterizzazione e ray-tracing.

Nel primo caso il calcolo è incentrato sugli oggetti presenti nella scena. Per ognuno di essi viene tracciato un raggio che parte da ogni vertice verso la telecamera.
Collegando i punti sulla griglia di visualizzazione, è possibile capire quale area dello schermo l'oggetto ricopre.
Dopodiché attraverso un buffer di profondità viene calcolato quali oggetti sono più vicino alla camera e quindi coprono gli oggetti più lontani.
Questo è un processo relativamente veloce perché il numero di raggi è limitato al numero di vertici presenti nella scena.
Di fatti, questo è il metodo che fino a qualche anno fa veniva usato in tutti i videogiochi per permettere una renderizzazione in tempo reale.
Questo spiega anche perché nei videogiochi ci sia la tendenza a tenere un numero di vertici molto più basso rispetto ad un film realizzato con la computer grafica.
Tuttavia, questo metodo presenta lo svantaggio di non riuscire a calcolare le luci in maniera realistica.

Il secondo metodo invece è incentrato sulla vista: dalla telecamera viene proiettato un raggio che interseca la griglia di visualizzazione per ogni pixel dell'immagine risultante.
Questi raggi, detti primari, continuano in linea retta finché non trovano un oggetto.
Dal punto di incidenza con l'oggetto, dipendentemente dalle proprietà fisiche di quest'ultimo, partono altri raggi, alcuni sono quelli secondari, che si dirigono direttamente verso le sorgenti luminose, per capire se l'oggetto è illuminato direttamente, oppure esiste un altro oggetto che proietta un ombra.
I restanti sono raggi di riflessione/rifrazione e permettono di calcolare la luce ambientale o indiretta che non era possibile calcolare nella rasterizzazione.
Per via dell'alto numero di computazioni necessarie, questo è un metodo molto più lento rispetto al primo, ma permette di raggiungere risultati molto più fotorealistici.

Recentemente comunque quest'ultimo metodo, è stato velocizzato a tal punto da poter essere implementato nei videogiochi, con una renderizazione in tempo reale.
Il primo invece è stato sviluppato a tal punto da ottenere immagini fotorealistiche, quasi quanto il secondo.
Blender propone due motori di renderizzazione, ognuno implementante uno di questi due metodi, rispettivamente Eevee e Cycles.

Nella realizzazione di questo cortometraggio è stato scelto di utilizzare Eevee come motore di renderizzazione. 
Il motivo di questa scelta deriva dal fatto che ciò ci ha permesso di accorciare notevolmente i tempi di rendering, al costo di un risultato meno realistico, che comunque non era uno dei requisiti. 
% Chapter 6

\chapter{Conclusioni} % Main chapter title

\label{Chapter6} % Change X to a consecutive number; for referencing this chapter elsewhere, use \ref{ChapterX}

%----------------------------------------------------------------------------------------
%	SECTION 1
%----------------------------------------------------------------------------------------

\section{Risultati}
\section{Lavori futuri}
 

%----------------------------------------------------------------------------------------
%	THESIS CONTENT - APPENDICES
%----------------------------------------------------------------------------------------

%\appendix % Cue to tell LaTeX that the following "chapters" are Appendices

% Include the appendices of the thesis as separate files from the Appendices folder
% Uncomment the lines as you write the Appendices

%\include{Appendices/AppendixTemplate}
%\include{Appendices/AppendixB}
%\include{Appendices/AppendixC}

%----------------------------------------------------------------------------------------
%	BIBLIOGRAPHY
%----------------------------------------------------------------------------------------

\printbibliography[heading=bibintoc]

%----------------------------------------------------------------------------------------

\end{document}  
